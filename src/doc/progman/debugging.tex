%
% PSI Programmer's Manual
%
% PSI Debugging Section
%
% TDC, 8 February, 1996
% Revised by TDC December 2002.
%
Debugging \PSIthree\ code using an interactive debugger, such as
\file{gdb} or \file{dbx} can be difficult at times because of the
complicated organization of this large program package.  This section
discusses some strategies and technical details of using such
debuggers with the \PSIthree\ code.  (Advice for debugging \PSIthree\
Fortran code may be found in section \ref{PSI_Fortran}.)

\subsection{Code Re-compilation}
Any section of \PSIthree\ code that needs to be debugged must first be
re-compiled with the ``-g'' flag turned on.  This flag is set in the
\file{MakeVars} file in the directory above each module or library's
source code directory.  For example, to turn on debugging in the
\PSIcscf\ program, one would first clean the existing object code out
of the \file{\$prefix/src/bin/cscf} directory using \file{make clean}.
Then edit \file{\$prefix/src/bin/MakeVars}, one directory above the
\PSIcscf\ source code: set \file{CDBG = -g} and, optionally \file{COPT
= } to turn off optimization flags.  (For modules using C++, the
analogous variables are \file{CXXDBG} and \file{CXXOPT}.  Then
re-compile the module.  If debugging information is needed for a
library routine as well, then follow this same procedure for the
library in question.  Technically, only the routines of interest need
to be re-compiled, though it is frequently more convenient to simply
re-compile the entire library or module.

\subsection{Multiple Source Code Directories}
The most difficult problem of debugging \PSIthree\ code is that object
code and source code generally reside in separate directories to allow
storage of objects for several achitectures simultaneously.  In
addition, library codes are kept separate from binary (module) codes.
If the code is compliled with \file{gcc/g++}, then this separation of
source and object code is of no consequence because the compiler
builds the full path to the source file directly into the object code.
However, for non-\file{gcc} compilations, one must know how to tell
the debugger where to find the sources.

Most interactive debuggers allow the programmer to specify multiple
source code search directories using simple command-line options.  For
example, if one were debugging the \PSIcscf\ program and needed
access to the \library{libciomr.a} library source code in addition to
that of \PSIcscf, one could use \file{gdb}'s ``dir'' command to search
several source code directories:
\begin{verbatim}
dir \$PSI/src/lib/libciomr
\end{verbatim}
Additionally, such commands can be placed in the user's
\file{\$HOME/.gdbinit} file.  In \file{dbx}, the ``use'' command
specifies multiple source directories.
