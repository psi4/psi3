\section{A \PSIthree\ Tutorial} \label{tutorial}

Before going into all the details of the \PSIthree\ input format, 
we will begin with a tutorial on how to run some very simple 
computations.

The first thing to point out is that electronic structure programs
like \PSIthree\ make significant use of disk drives.  Therefore, 
for any but the smallest calculations, {\em it is very important for the user 
to be sure that \PSIthree\ is writing its temporary files to a 
disk drive physically attached to the computer running the computation}.
If a user's directory is remotely mounted by NFS, and if \PSIthree\
tries to write its temporary files to this directory, it will slow 
the program and the network down dramatically.  To avoid this
situation, by default \PSIthree\ will write temporary files to 
\file{/tmp}.  This will work for the small examples in this tutorial,
but for real computations you will want to set up a default scratch
file path as described in sections \ref{scratchfile} and \ref{psirc},
since your \file{/tmp} directory may not be large enough.

The \PSIthree\ suite of codes is built around a modular design which 
allows it great power and flexibility. Any module can be run 
independently (provided suitable datafiles, of course). There also exists a
master program, appropriately called \PSIdriver, which will parse 
an input file, recognize the overall calculation desired, 
and run all the necessary modules in the correct order.  As we will
see later, it is possible to define entirely new calculation types simply 
by editing a text file.

By default, the \PSIdriver\ program reads its input from a text file
called \inputdat, and output from \PSIthree\ is written to 
\outputdat.  However, it is also possible to specify different input
and output filenames by adding these to the command line, like this:
\begin{verbatim}
  psi3 input-name output-name
\end{verbatim}

What does a \PSIthree\ input file look like?  First of all, due to
the modular nature of the program, the input may be split into different
sections, each beginning with a keyword and enclosed in parentheses.
For now, we will ignore this flexibility and simply put all of the 
input into a default area called \keyword{psi}:

\begin{verbatim}
% This is a sample PSI3 input file.
% Anything after a percent sign is treated as a comment.

psi: (
  label = "cc-pVDZ SCF H2O"
  jobtype = sp
  wfn = scf
  reference = rhf
  basis = "cc-pVDZ"
  puream = true
  zmat = (
    o
    h 1 0.957
    h 1 0.957 2 104.5
  )
)
\end{verbatim}

As you can see, the input is made up of \keyword{keyword = value} pairs.  
Values can be strings (like \keyword{"cc-pVDZ"}), booleans 
(\keyword{true/false, 1/0, yes/no}), integers, or real numbers.  One can 
also have arrays of data (as in the the \keyword{zmat} entry).  Generally
speaking, input is not case-sensitive. 

The above input is for a Hartree-Fock SCF (\keyword{wfn = scf})
calculation of water using a cc-pVDZ basis set 
(\keyword{basis = "cc-pVDZ"}).  This is a 
single-point calculation (\keyword{jobtype = sp}) at the experimental 
geometry.  Since the ground state of water is closed shell, restricted 
Hatree-Fock (RHF) orbitals are used (\keyword{reference = rhf}).  
The geometry is specified in the \keyword{zmat} array using the
standard quantum chemistry Z-matrix input format (for beginners, a short
tutorial on Z-matrix inputs is available at
\htmladdnormallink{{\tt www-rcf.usc.edu/$\sim$krylov/AbInitioCourse/zmat.html}}{
http://www-rcf.usc.edu/~krylov/AbInitioCourse/zmat.html}).  Geometries
may also be input in Cartesian coordinates, as we will see later.

The final keyword, \keyword{puream}, tells the program whether or not
to use Cartesian basis functions, or basis functions derived from the
pure angular momentum functions.  For $d$ orbitals, for example, 
there are 6 Cartesian $d$ functions, but only 5 pure angular 
momentum $d$ functions.  Dunning's correlation consistent basis sets
(cc-pVXZ, etc.) were designed to use pure angular momentum functions
(\keyword{puream=true}), and other basis sets may have different 
``standard choices'' for using pure or Cartesian functions.  The \PSIthree\ 
program defaults to \keyword{puream = false} unless told otherwise.

Why are some strings in quotation marks (e.g., \keyword{label = "cc-pVDZ
SCF H2O"}), while others are not (e.g., \keyword{wfn = scf})?  Any
string containing spaces or other special characters like asterisks
must be placed in double quotes.  Since basis sets often contain special
characters, it is a good idea to specify the basis in quotes always.
For values which don't have to be put in double quotes, it never hurts
to put them in quotes anyway.

Let's run this calculation.  Assuming the input is given in a file
\file{sp.in}, we can run \PSIthree\ as:
\begin{verbatim}
psi3 sp.in sp.out
\end{verbatim}

This will run the driver program \PSIdriver\, which will read the
input from \file{sp.in} and write the output to \file{sp.out}.  
The driver program will print something like this:
\begin{verbatim}
 The PSI3 Execution Driver
 
PSI3 will perform a RHF SCF energy computation.
 
The following programs will be executed:
 
 input
 cints
 cscf
 
input
cints
cscf
\end{verbatim}
The \PSIdriver\ program figures out what type of calculation it is
(a restricted Hartree-Fock self-consistent-field energy computation)
and it runs the appropriate \PSIthree\ modules, which are 
\PSIinput, \PSIcints, and \PSIcscf.  The \PSIinput\ program reads
the geometry and basis set information from the input file and
places this information in a ``checkpoint file'' (with a filename
ending in \file {.32}) which is used
during the calculation.  The \PSIcints\ module computes the one- and
two-electron integrals, and the \PSIcscf\ module computes the 
SCF energy.  Since this is all we requested, the program stops here.

What does the output of the program look like?  There is quite a lot
in the output, so we will focus only on the most pertinent portions.
The first part should look something like this:
\begin{verbatim}
******************************************************************************
tstart called on aurelius.chemistry.gatech.edu
Tue Aug 26 17:35:10 2003
                                                                                
                                --------------
                                  WELCOME TO
                                    PSI  3
                                --------------
                                                                                
  LABEL       = cc-pVDZ SCF H2O
  SHOWNORM    = 0
  PUREAM      = 1
  PRINT_LVL   = 1
                                                                                
  -Geometry before Center-of-Mass shift (a.u.):
       Center              X                  Y                   Z
    ------------   -----------------  -----------------  -----------------
          OXYGEN      0.000000000000     0.000000000000     0.000000000000
        HYDROGEN      0.000000000000     0.000000000000     1.808467771070
        HYDROGEN      1.750863805261     0.000000000000    -0.452804167853
                                                                                
...
\end{verbatim}

THIS JUNK BELOW IS THE OLD STUFF, KEEPING IT IN TEMPORARILY TO PLUCK
BITS OUT OF IT. ---CDS

This will run \PSIthree\, which will read the input from 
will get the online manual page for the \PSIdriver\ master program.
Other man pages for the \PSIdriver\ code can be retrieved likewise.

The only required file is to run \PSIthree\ is \inputdat. 
Unfortunately, no good and simple description of this
file exists. It is rather simple though, and essentially
free format. Examples can be found in
almost every man page. It should be fairly easy to
understand all the examples using common sense. 

\subsection{Format of \inputdat}
Some keys to remember are the following: 
\begin{itemize}
\item Input parsing is case insensitive; once parsed, everything is treated as upper case. 
\item Anything following a percent mark, i.e. \%, is commented out up until the next carriage return. 
\item White space of more than a single space is ignored. 
\item Anything between double quotes, e.g. "/usr/c4/sw/psi", 
is considered one token; there is no change of case, and special 
characters and white space are maintained as part of the token 
but  otherwise ignored. 
\item The old (Fortran) input parser does not like the TAB character.
Please avoid using Tabs until this goes away.  There does not appear
to be a problem with the C library (libipv1).
\end{itemize}
Input data types are the following: 
\begin{itemize}
\item String : a character sequence 
\item Integer : a \celem{sizeof(integer)} byte integer datum 
\item Real : a \celem{sizeof(double)} byte real datum 
\item Boolean : \keyword{yes}, \keyword{true}, \keyword{1} -- these three values are equivalent;
\keyword{no}, \keyword{false}, \keyword{0} -- these three values are equivalent 
\end{itemize}
Input parameters come in only a few flavors: 
\begin{itemize}
\item \keyword{Keywords = value}, e.g. \keyword{convergence = 12} 
\item \keyword{Vectors = (value1 value2 ... valuen)}, e.g. \keyword{docc = (2 0 1 1)}
\item \keyword{Arrays = ((i1 i2 ... in) (j1 j2 ... jn) ... (n1 n2 ... nn))}
Note the proper number of opening and closing parentheses. Arrays can 
conceptually be any depth/dimension but in practice never go
much beyond level 2. The elements of vectors and arrays 
need not be of the same data type. 
\end{itemize}

Segmented Keywords are joined by a colon, i.e. ":". 
\begin{verbatim}
  scf:convergence = 12
  gugaci:convergence = 8
\end{verbatim}
Segmented keywords with common initial segments can be joined in vectors. 
\begin{verbatim}
  scf:(convergence = 12
       docc = (2 0 1 1))
\end{verbatim}
Note the equivalence of the following three examples with the preceding one. 
\begin{verbatim}
  scf:(convergence = 12 docc = (2 0 1 1)
  scf:( convergence=12  docc=(2 0 1 1))
  scf: (
     convergence = 12 
     docc = (2 0 1 1)
  )
\end{verbatim}

A "default" token can be used as a wild card, to fill any initial 
token field. However, explicit tokens will always override default
tokens if both are present. For instance, a typical keyword search
priority would be: First 
\begin{verbatim}
  module_id:parameter
\end{verbatim}
and then 
\begin{verbatim}
  default:parameter
\end{verbatim}
Thus the two following examples will achieve the same effect when
the module \PSItransqt\ is run, but not when \PSIcscf\ is run. 
\begin{verbatim}
  default: (
    convergence = 8
    docc = (2 0 1 1)
  )
  scf: (
    convergence = 12
  )

  default: (
    convergence = 8
    docc = (2 0 1 1)
  )
  scf: (
    convergence = 10
  )
\end{verbatim}

Ok, enough of all that, you've got the basics down,
the rest can be learned by doing. The smallest \inputdat\
file you would probably ever want to use, and this only as
a learing experience, is: 
\begin{verbatim}
  % file input.dat
  psi: (
    check = true
    )
  % end file input.dat
\end{verbatim}
If we run psi now, what happens? 
\begin{verbatim}
  c4-20:> psi
                      The Psi Execution Driver
   ERROR: a problem arose while reading the required string valued keyword 'WFN'
  c4-20:>
\end{verbatim}
Obviously we need to add a \keyword{WFN}. So 
\begin{verbatim}
  % file input.dat
  psi: (
    check = true
    wfn = scf          % for instance
    )
  % end file input.dat

  c4-20:> psi
                      The Psi Execution Driver
   WFN       = SCF
   DERTYPE   = NONE
   REFERENCE = RHF
   CHECK     = YES
   
   'CHECK' is YES, so nothing will be executed.
   The following programs would otherwise be executed:
   cints
   cscf
  c4-20:>
\end{verbatim}
As you can see, some defaults are chosen. We are now doing a
RHF SCF Energy Point. First the module (or program) \PSIcints\ will
run to calculate the integrals, then \PSIcscf\ will run to calculate the
RHF SCF energy. 

Should we run this, just for the learning experience? Sure,
but first we need to do a couple things. Like find out what
files and input parameters are needed by \PSIcints\ and \PSIcscf. 
\begin{verbatim}
  c4-20:> man cints
  c4-20:> man cscf
\end{verbatim}
\PSIcints\ needs \inputdat\ and \FILE30\ and will generate \FILE33,
\FILE35, \FILE36, and \FILE37. It requires only 2 input parameters
and the defaults look ok. \PSIcscf\ needs \inputdat, \FILE30, \FILE33,
\FILE35, \FILE36, and \FILE37 and
a few input parameters which we will come back to. 

So what's this \FILE30? It is the basic binary working file for the \PSIthree\
quantum chemistry package. It contains all the juicy details of your job,
such as geometry, basis set, occupation scheme, etc.
Since no one likes to write binary files,
we will create \FILE30 with the program \PSIinput.
For more information on this program 
\begin{verbatim}
  c4-20:> man input
\end{verbatim}


\subsection{Setting up a Calculation}
I've decided that we'll start with STO water for our example, and
after reading the man pages described above I come up with the following: 
\begin{verbatim}
  % file input.dat
  default: (

    label = "water STO HF energy point"
    memory = (8.0 Mbytes)
    wfn = scf
    reference = rhf

    files: (
      default: (
        name = "h2osto"
        nvolume = 4
        volume1 = "/tmp1/psiuser/"
        volume2 = "/tmp2/psiuser/"
        volume3 = "/tmp2/psiuser/"
        volume4 = "/tmp2/psiuser/"
        )
      file30: (
        nvolume = 1
        volume1 = "./"
        )
      )
    )

  psi: (
    check = true
    )

  input: (
  %    note that all atoms are specified now,
  %    not only the symmetry unique portion
    basis = sto-3g
    units = angstrom
    zmat = (
      (o)
      (h 1 0.9600)
      (h 1 0.9600 2 104.5)
      )
    )
  % end file input.dat
\end{verbatim}
You may wonder about the memory flag I've tucked in the default. As far
as I know, it's undocumented, but can be set for some PSI modules.
Acceptable units are REAL, INTEGER, BYTES,
KBYTES, or MBYTES. Generally it is a good practice to keep
\keyword{default:memory} small, and 
increase \keyword{program\_id:memory} as necessary. 

The \keyword{default:files} section tells \PSIthree\ where to
store temporary (scratch) files.  Here, a user is telling \PSIthree\ to 
write the scratch files to four different directories (where in this
case ``psiuser'' is used as the user's username).
The file-handling capability of \PSIthree\
is very flexible --- it is possible, for example, to tell each module to
place files in different locations (although this would rarely ever be
useful).  By placing a \keyword{files} section in the \keyword{default}
area, \PSIthree\ will use this information by default for all of its
modules unless overridden by a \keyword{files} section inside a particular
module (like \PSIcscf, for example).  Specifying four volumes, such as in
this example, tells \PSIthree\ to stripe (split the files) over four 
different directories.  This can be useful if there are four hard disks, all
used for scratch data.  However, modern operating systems allow different
physical hard drives to be striped automatically by the operating system
into a single directory, so having \keyword{nvolume} more than 1 is pretty
unusual nowadays.  

One downside to listing these specific scratch directories is that, if you
send your \PSIthree\ input file to a co-worker, he or she will have to
modify the file to use their scratch directories instead of yours (assuming
you don't all write to the same directory, like \file{/tmp}).  This problem
is solved by the \file{.psirc} file.  If this file exists
in the user's home directory, then it will override the \keyword{file}
information in \file{input.dat}.  For example, with a \file{.psirc} file 
containing the following information,

\begin{verbatim}
default: (
    files: (
      default: (
        nvolume = 4
        volume1 = "/tmp1/psiuser/"
        volume2 = "/tmp2/psiuser/"
        volume3 = "/tmp2/psiuser/"
        volume4 = "/tmp2/psiuser/"
      )
      file30: (
        nvolume = 1
        volume1 = "./"
      )
    )
)
\end{verbatim}
then the relevant section in \file{input.dat} can be vastly simplified
to just
\begin{verbatim}
  files: (
    default: ( name = "h2o" )
    )
\end{verbatim}

Once you have a \file{input.dat} file prepared, the first thing to do is
to create the checkpoint file, the previously mentioned \FILE30.  You
can do this by running the program
\begin{verbatim}
  c4-20:> input
\end{verbatim}
You should now have two new files \outputdat\ and \keyword{h2osto.30}.
The latter file is a binary file, it cannot be directly examined.  The first
file should contain text giving the program output.  Take a look at this
file and see what kind of information it gives.

Going on, 
\begin{verbatim}
  c4-20:> psi
\end{verbatim}
reminds us that we will be running \PSIcints\ and \PSIcscf. We have everything
we need for the first, but how about the second. 
\begin{verbatim}
  c4-20:> man cscf
\end{verbatim}
Well, there are alot of possibilities, but the defaults are generally
sound, so we really only need to add in our occupation scheme for
the doubly-occupied orbitals. Or do we? The fact is
\PSIcscf\ can guess occupations for you using
core Hamiltonian orbital eigenvalues, but, as with any computer
program, you have to be cautious. Let us be adventurous here and
let the program guess.
In general, always make sure you are computing the state you
want to and not the state that the program chose. 

Notice at this point that I am leaving out the
majority of the file \inputdat. It is getting a little big
to repetitiously include it within this document. 

\subsection{Carrying Out a Calculation}
Assuming you have done everything detailed in the previous
section, we can run an actual job and get a number out. 
\begin{verbatim}
  c4-20:> psi
\end{verbatim}
Ahhhh. We forgot to change the \keyword{psi:check=true} flag,
so nothing was really run. Change it to \keyword{psi:check=false},
or simply comment it out (since the default value is false) by inserting a "\keyword{ \%}" at
the start of that line, and run psi again. Take a look at the file \outputdat\
and become familiar with it. 

We now have the basic \inputdat\ file. Only minor modifications will allow
it to be used to run a great variety of jobs. First
of all let's move on to another basis set. The only thing that needs to be
changed is the parameter \keyword{input:basis=sto-3g} to,
for instance, \keyword{input:basis=dz}. But for consistency,
why not change \keyword{default:label} and \keyword{default:files:default:name}? 
\begin{verbatim}
  c4-20:> input
  c4-20:> psi
\end{verbatim}
Whenever you change the basis parameter, you must run input so that
the change in basis is included into \FILE30. Very important.
Changing \keyword{default:files:default:name} every time you
change the basis parameter is a good habit, as it will
insure that you run \PSIinput. Do you see why? 

An aside about \PSIcscf\ and guess wavefunctions: when you run \PSIcscf,
it (by default, but can be overridden) automatically checks
to see if there is an old wavefunction in \FILE30\ which can be used
as a guess wavefunction, and if so, it uses it.
Each time \PSIcscf\ completes, (converged or not) it writes
the final wavefunction to \FILE30. Each time \PSIinput\ is run, it overwrites
\FILE30\ and any wavefunction that might have been in there is
lost. Nothing tricky about any of this, but just something to
keep in mind.  If you want to re-use MO's from a previous calculation as a
guess, you can run \PSIinput\ with the {\tt --chkptmos} option.  This will
work even if the previous calculation used a different basis set, but the
geometry and point group symmetry should be the same.  Another useful
flag is {\tt --noreorient}, which prevents the molecule from reorienting
after shifting the origin to the center-of-mass.

Well, energy points are well and good, but without optimizations,
we are not going to get very far. So, we need \keyword{psi:dertype=first},
\keyword{psi:opt=true}, and \keyword{psi:nopt=7} to run up to 7 cycles of geometry
optimization (Or we could set \keyword{default:dertype=first}
and \keyword{default:opt=true}, or even mix and match). Set \keyword{psi:check=true}
and run \PSIdriver\ to see what modules will be run. Then check the man
pages for each module and see what you need for it to run. For instance,
in this case we see that \PSIcderiv\ will be run. {\tt man \PSIcints}
shows that us that everything is set OK. \PSIoptking\ is the
last program to be run in each optimization cycle. \PSItwo\
users will appreciate the ease with which it can optimize molecules
as it can generate internal coordinates for you as well as guess a
force constant matrix. Hence defaults are sufficient to run this program.

OK! Go ahead and run \PSIdriver. In the 5th cycle \PSIoptking\ 
should return a non-zero value and \PSIdriver\ will stop the
procedure. It will indicate that the optimization is
over. Let's up the ante and go for an optimization with a DZP basis set.

You know the procedure for improving the basis set.
Just switch the label, the files name and the basis parameter.
Oh, before I forgot, you have to remove file named \optaux. Then run \PSIinput\ 
and \PSIdriver. The job runs fine. Go ahead and look at \outputdat.
You may also wish to look at \FILE{11.dat}. \fconstdat\ contains now
an improved force constant matrix. \PSIoptking\ will automatically
use this improved FC matrix if it is present.

OK, the calculation did converge in 4 steps. What if it didn't?
Just increasing the number of optimization steps might just
do that but might not. In general, 10 optimization steps should be enough
for anyone. If it doesn't converge by then,
your initial guess geometry or initial FC matrix
or both were bad. Rethink your situation.
Oops, this is supposed to be a howto-psi, not a QC-theology. 
