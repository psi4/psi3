\PSIthree\ input files are case-insensitive and free-format, with
a grammar designed for maximum flexibility and relative simplicity.
Input values are assigned using the structure:
\begin{verbatim}
keyword = value
\end{verbatim}
where {\tt keyword} is the parameter chosen (e.g., {\tt convergence})
and {\tt value} has one of the following data types:
\begin{itemize}
\item string: A character sequence surrounded by double-quotes.
  Example: {\tt basis = "cc-pVDZ"}
\item integer: Any positive or negative number (or zero) with no
  decimal point.  Example: {\tt maxiter = 100}
\item real: Any floating-point number.  Example: {\tt omega = 0.077357}
\item boolean: {\tt true}, {\tt false}, {\tt yes}, {\tt no}, {\tt 1},
  {\tt 0}.
\item array: a parenthetical list of values of the above data types.
  Example: {\tt docc = (3 0 1 1)}.  
\end{itemize}
Note that the input parsing system is general enough to allow
multidimensional arrays, with elements of more than one data type.  A
good example is the z-matrix keyword:
\begin{verbatim}
zmat = (
  (o)
  (h 1 r)
  (h 1 r 2 a)
)
\end{verbatim}

Keywords are generally grouped together in common segments, the most
important of which is the {\tt psi:} segment.  For example, a simple
\PSIthree\ input file for a single-point 

grouped together under a single 


Some additional aspects of the \PSIthree\ grammar to keep in mind:
\begin{itemize}
\item The ``\%'' character denotes a comment line, i.e. any
  information following the ``\%'' up to the next linebreak is ignored
  by the program.
\item Anything in between double quotes (i.e. strings) is case-sensitive.
\item Multiple spaces are treated as a single space.
\end{itemize}
