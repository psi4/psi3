\PSIthree\ input files are case-insensitive and free-format, with
a grammar designed for maximum flexibility and relative simplicity.
Input values are assigned using the structure:
\begin{verbatim}
keyword = value
\end{verbatim}
where {\tt keyword} is the parameter chosen (e.g., {\tt convergence})
and {\tt value} has one of the following data types:
\begin{itemize}
\item string: A character sequence surrounded by double-quotes.
  Example: {\tt basis = \"cc-pVDZ\"}
\item integer: Any positive or negative number (or zero) with no
  decimal point.  Example: {\tt maxiter = 100}
\item real: Any floating-point number.  Example: {\tt omega = 0.077357}
\item boolean: {\tt true}, {\tt false}, {\tt yes}, {\tt no}, {\tt 1},
  {\tt 0}.
\item array: a parenthetical list of values of the above data types.
  Example: {\tt docc = (3 0 1 1)}.  
\end{itemize}
Note that the input parsing system is general enough to allow
multidimensional arrays, with elements of more than one data type.  A
good example is the z-matrix keyword:
\begin{verbatim}
zmat = (
  (o)
  (h 1 r)
  (h 1 r 2 a)
)
\end{verbatim}

Keywords must grouped together in blocks, based on the module or
modules that require them.  The default block is labelled {\tt psi:},
and most users will require only a {\tt psi:} block when using
\PSIthree.  For example, the following is a simple input file for a
single-point CCSD energy calculation on H$_2$O:
\begin{verbatim}
psi: (
  label = "6-31G**/CCSD H2O"
  wfn = ccsd
  reference = rhf
  jobtype = sp
  basis = "6-31G**"
  zmat = (
    (o)
    (h 1 r)
    (h 1 r 2 a)
  )
  zvars = (
    (r 1.0)
    (a 104.5)
  )
)
\end{verbatim}
In this example, the {\tt psi:} identifier collects all the keywords
(of varying types) together.  Every \PSIthree\ module will have access
to every keyword in the {\tt psi:} block by default.  One may use
other identifiers (e.g., {\tt ccenergy:}) to separate certain keywords
to be used only by selected modules.  For example, consider the
keyword {\tt convergence}, which is used by several \PSIthree\ modules
to determine the convergence criteria for constructing various types
of wave functions.  If one wanted to use a high convergence cutoff for the
\PSIthree\ SCF module but a lower cutoff for the coupled cluster
module, one could use the following input:
\begin{verbatim}
psi: (
  label = "6-31G**/CCSD H2O"
  wfn = ccsd
  reference = rhf
  jobtype = sp
  basis = "6-31G**"
  zmat = (
    (o)
    (h 1 r)
    (h 1 r 2 a)
  )
  zvars = (
    (r 1.0)
    (a 104.5)
  )
  convergence = 7
)

scf:convergence = 12
\end{verbatim}
Note that, since we have only one keyword associated with the {\tt
  scf:} block, we do not need to enclose it parentheses.

Some additional aspects of the \PSIthree\ grammar to keep in mind:
\begin{itemize}
\item The ``\%'' character denotes a comment line, i.e. any
  information following the ``\%'' up to the next linebreak is ignored
  by the program.
\item Anything in between double quotes (i.e. strings) is case-sensitive.
\item Multiple spaces are treated as a single space.
\end{itemize}

\subsection{Specifying Scratch Disk Usage in \PSIthree: The {\tt
    files:} Section of Input}

Depending on the calculation, the \PSIthree\ package often requires
substantial temporary disk storage for integrals, wave function
ampltiudes, etc.  To allow for various customized arrangements of
scratch disks, the \PSIthree\ {\tt files:} block gives the user
considerable control over how temporary files are organized, including
file names, scratch directories, and the ability to ``stripe'' files
over several disks (much like RAID0 systems).  This section of
keywords is normally placed within the {\tt psi:} section of input,
but may be used for specific \PSIthree\ modules, just like other
keywords.

For example, if the user is working with \PSIthree\ on a computer
system with only one scratch disk (mounted at, e.g., {\tt /scr}), one
could identify the disk in the input file as follows:
\begin{verbatim}
psi: (
  label = "6-31G**/CCSD H2O"
  wfn = ccsd
  reference = rhf
  jobtype = sp
  basis = "6-31G**"
  zmat = (
    (o)
    (h 1 r)
    (h 1 r 2 a)
  )
  zvars = (
    (r 1.0)
    (a 104.5)
  )

  files: (
    default: (
      nvolume = 1
      volume1 = "/scr/"
    )
  )
)
\end{verbatim}
The {\nvolume} keyword indicates the number of scratch
directories/disks to be used to stripe files, and each of these is
specified by a corresponding {\tt volumen} keyword.  (NB: the trailing
slash ``/'' is essential in the directoy name.)  Thus, in the above
example, all temporary storage files generated by the various
\PSIthree\ modules would automatically be placed in the {\tt /scr}
directory.  

By default, the scratch files are given the prefix ``{\tt psi}'', and
named ``{\tt psi.nnn}'', where {\tt nnn} is a number used by the
\PSIthree\ modules.  The user can select a different prefix by
specifying it in the input file with the {\tt name} keyword:
\begin{verbatim}
psi: (
  files: (
    default: (
      name = "H2O"
      nvolume = 1
      volume1 = "/scr/"
    )
  )
)
\end{verbatim}
The {\tt name} keyword allows the user to store data associated with
multiple calculations in the same scratch area.  Alternatively, one
may specify the filename prefix on the command-line of the {\tt psi3}
driver program (or any \PSIthree\ module) with the {\tt --p} argument:
\begin{verbatim}
psi3 -p H2O
\end{verbatim}

If the user has multiple scratch areas available, \PSIthree\ files may
be automatically split (evenly) across them:
\begin{verbatim}
psi: (
  files: (
    default: (
      nvolume = 3
      volume1 = "/scr1/"
      volume2 = "/scr2/"
      volume3 = "/scr3/"
    )
  )
)
\end{verbatim}
In this case, each \PSIthree\ datafile will be written in chunks (65
kB each) to three separate files, e.g., {\tt /scr1/psi.72}, {\tt
/scr2/psi.72}, and {\tt /scr3/psi.72}.  The maximum number of volumes
allowed for striping files is eight (8), though this may be easily
extended in the \PSIthree\ I/O code, if necessary.

\subsection{The {\tt .psirc} File}

