This manual is intended to provide users with up-to-date information
on running the \PSIthree suite of {\em ab initio} quantum chemical
programs.  In section \ref{preliminary} we provide an overview of some
of the features of \PSIthree\ along with the prerequisite steps for
running calculations.  Section \ref{tutorial} provides a brief
tutorial to help new users get started.  Section \ref{running} offers
further details into the structure of \PSIthree\ calculations,
including descriptions of the most important options and modules.
Later sections deal with more advanced aspects of \PSIthree\
calculations, including cutomization of basis sets, scratch disk
space, and troubleshooting.  The appendix includes a description of
the input keywords and command-line options for each module, as well
as numerous examples of \PSIthree\ input and basis set files.

Instructions for obtaining, compiling, and installing \PSIthree\ on
UNIX systems are given in the separate Installation Manual.  The
latest \PSIthree\ documentation can be found at
\htmladdnormallink{{\tt http://vergil.chemistry.gatech.edu/psi/}}
{http://vergil.chemistry.gatech.edu/}


