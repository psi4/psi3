%
% PSI Programmer's Manual
%
% Fundamental PSI Functions Intro.
%
% Daniel Crawford, 26 January, 1996
%

Every \PSIthree\ module (e.g. \PSIcscf) must perform two specific tasks,
regardless of the individual module's specific purpose(s).  These are: (1)
obtaining user input options, and (2) writing to and reading from binary
files (e.g. \FILE{30}).  \PSIthree\ programs written in the C programming
language make use of three libraries which provide all the tools necessary to
carry out these functions efficiently :
\begin{itemize}
\item \library{libipv1.a} (version 1 of the \PSIthree\ input parser);
\item \library{libciomr.a} (the \PSIthree\ old-style C-language I/O and math
routine library);
\item \library{libpsio.a} (the \PSIthree\ new table-of-content-based
C-language I/O library).
\end{itemize}
In the next two sections we will describe these libraries and some of
their most important functions.  We will then discuss
the most basic components of a \PSIthree\ C-language program.

\subsection{The Input Parser}\label{C_IP}
%
% PSI Programmer's Manual
%
% Input Parsing Section
%
% Justin T. Fermann, 1 February 1996
%
% Updated and improved(?) by Edward F. Valeev, 7 June 2000
%
The input parsing library is built for the purpose of reading in the
contents of an input file with the syntax of \inputdat\ and storing
the contents specific to certain keywords supplied. To perform such a
task \library{libipv1.a} has three parts: (1) the parser; (2) the
lexical scanner; (3) keyword storage and retrieval.

The format of \inputdat\ follows certain rules which should probably
referred to as the PSI input grammar. There is a description of most
of those rules in \PSIthree\ User's Manual. A complete definition of
the PSI input grammar is encoded in \file{parse.y} (see below).  To
read a grammar we need a parser -- the first component of
\library{libipv1.a}. Then the identified lexical elements of
\inputdat\ (keywords and keyword values) need to be scanned for
presence of ``forbidden'' characters (e.g.\ a space may not be a part
of a string unless the string is placed between parentheses).  This
task is performed by the lexical scanner --- the second component of
\library{libipv1.a}. Finally, scanned-in pairs of keyword-value(s) are
stored in a hierarchical data structure (a tree). When a particular
option is needed, the set of stored keywords and values is searched
for the one queried and the value returned.  In this way, options of
varying type can be assigned, i.e.\ rather than having a line of
integers, each corresponding to a program variable, mnemonic character
string variables can be parsed and interpreted into program variables.
It's also easier to implement default options, allowing a more spartan
input deck.  The set of input-parsing routines in \library{libipv1.a}
is really not complicated to use, but the manner in which data is
stored is somewhat painful to grasp at first.

The following is a list of the names of the individual source files in
\library{libipv1} and a summary of their contents.  After that is a
list of the syntax of specific functions and their use.  Last is a
simple illustration of the use of this library, taken mostly from
\PSIcscf.

\subsubsection{Source Files}

\begin{itemize}
\item Header files
  \begin{itemize}
  \item \file{ip\_error.h} Defines for error return values.
  \item \file{ip\_global.h} cpp macros to make Curt happy.
  \item \file{ip\_lib.h} \#include's everything.
  \item \file{ip\_types.h} Various structures and unions specific to
                          \library{libipv1}.
  \end{itemize}
\item Other Source
  \begin{itemize}
  \item \file{parse.y} Yacc source encoding the PSI input grammar.
  Read by {\tt yacc} (or {\tt bison}) -- a parser generator program.
  \item \file{scan.l} Lex source describing lexical elements allowed
  in \inputdat. Read by {\tt lex} (or {\tt flex}) -- a lexer generator
  program. 
  \item \file{*.gbl, *.lcl} cpp macros to mimic variable argument lists.
  \end{itemize}
\item C source
  \begin{itemize} 
  \item \file{ip\_alloc.c} Allocates keyword tree elements.
  \item \file{ip\_cwk.c} Routines to manipulate the current working
                       keyword tree.
  \item \file{ip\_data.c} Routines to handle reading of arrays and
                         scaler keyword assignments in input.
  \item \file{ip\_error.c} Error reporting functions.
  \item \file{ip\_karray.c} Other things to deal with keyword arrays.
  \item \file{ip\_print.c} Routines to print sections of the keyword
  tree.
  \item \file{ip\_read.c} All the file manipulation routines.  Reading
                  of \inputdat\ and building the keyword tree from
                  which information is later plucked.
  \end{itemize}
\end{itemize}

\subsubsection{Syntax}

\begin{center} \file{ip\_cwk.c}\\ \end{center}

\celem{void ip\_cwk\_clear();} \\
Clears current working keyword.  Used when initializing input or switching
from one section to another (:DEFAULT and :CSCF to :INTCO, for instance).

\celem{void ip\_cwk\_add(char *kwd);} \\
Adds \celem{kwd} to the list of current working keywords.  Allows parsing of 
variables under that keyword out of the input file (files) which has
(have) been read or will be read in the future using \celem{ip\_append}.
The keyword \celem{kwd} can only be removed from the list of current working
keywords by purging the entire list using \celem{ip\_cwk\_clear}. 

\begin{center} \file{ip\_data.c} \\ \end{center}

\celem{int ip\_count(char *kwd, int *count, int n);} \\
Counts the elements in the n'th element of the array \celem{kwd}.

\celem{int ip\_boolean(char *kwd, int *bool, int n);} \\
Parses n'th element of \celem{kwd} as boolean (true, 1, yes; false, 0, no)
into 1 or 0 returned in \celem{bool}.

\file{int ip\_exist(char *kwd, int n);} \\
Returns 1 if n'th element of \celem{kwd} exists.  Unfortunately, n must be 0.

\file{int ip\_data(char *kwd, char *conv, void *value, int n 
      [, int o1, ..., int on]);} \\
Looks for keyword \celem{kwd}, finds the value associated with it,
converts it according to the format specification given in
\celem{conv}, and stores the result in \celem{value}.  Note that
\celem{value} is a \celem{void *} so this routine can handle any data
type, but it is the programmer's responsibility to ensure that the
pointer passed to this routine is of the appropriate pointer type for
the data.  The value found by the input parser depends on the value of
\celem{n} and any optional additional arguments.  \celem{n} is the
number of additional arguments.  If \celem{n} is 0, then there are no
additional arguments, and the keyword has only one value associated
with it.  If the keyword has an array associated with it, then
\celem{n} is 1 and the one additional argument is which element of the
array to pick.  If \celem{kwd} specifies an array of arrays, then
\celem{n} is 2, the first additional argument is the number of the
first array, and the second argument is the number of the element
within that array, etc.  Deep in here, the code calls a
\celem{sscanf(read, conv, value);}, so that's the real meaning of
variables.

\celem{int ip\_string(char *kwd, char **value, int n, [int o1, ..., int on]);}\\
Parses the string associated with \celem{kwd} stores it in \celem{value}.
The role of \celem{n} and optional arguments is the same as that
described above for \celem{ip\_data()}.

\celem{int ip\_value(char *kwd, ip\_value\_t **ip\_val, int n);} \\
Grabs the section of keyword tree at \celem{kwd} and stores it in 
\celem{ip\_val}
for the programmer's use - this is usually not used, since you need to 
understand the structure of \celem{ip\_value\_t}.

\celem{int ip\_int\_array(char *kwd, int *arr, int n);} \\
Reads n integers into array \celem{arr}.

\begin{center} \celem{ip\_read.c} \\ \end{center}

\celem{void ip\_set\_uppercase(int uc);} \\
Sets parsing to case sensitive if uc==0, I think.

\celem{void ip\_initialize(FILE *in, FILE *out);} \\
Calls \celem{yyparse();} followed by \celem{ip\_cwk\_clear();} followed by 
\celem{ip\_internal\_values();}.  This routine reads the entire input deck
and stores it into the keyword tree for access later.

\celem{void ip\_append(FILE *in, FILE *out);} \\
Same thing as \celem{ip\_initialize();}, except this doesn't clear the 
\celem{cwk} first.  Used for parsing another input file, such 
as \celem{intco.dat}.

\celem{void ip\_done();} \\
Frees up the keyword tree.

\begin{center} \celem{ip\_read.c} \\ \end{center}

\celem{void ip\_print\_tree(FILE *out, ip_keyword_tree_t *tree);} \\
Prints out \celem{tree} to \celem{out}. If \celem{tree} is set to \celem{NULL},
then the current working keyword tree will be printed out.
This function is useful for debugging problems with parsing.

\subsubsection{Sample Use from \PSIcscf}
These are two slightly simplified pieces of actual code.

From \file{cscf.c}:
\begin{verbatim}
#include <libipv1/ip_lib.h>

   ffile(&infile,"input.dat",2);     /* input and output files. */
   ffile(&outfile,"output.dat",1);   /* call them whatever you want. */

   ip_set_uppercase(1);              /* case sensitivity selection */
   ip_initialize(infile,outfile);    /* reads input.dat and stores it all */

   ip_cwk_add(":DEFAULT");           /* adds default section */
   ip_cwk_add(":SCF");               /* adds scf section */

   ip_string("OUTPUT",&output,0);    /* bet you didn't know you could */
   if(!strcmp(output,"TERMINAL")) {  /* have cscf write to stdout!    */
     outfile = stdout;
     }
   else if(!strcmp(output,"WRITE")) {
     fclose(outfile);
     ffile(&outfile,"output.dat",0);
     }
\end{verbatim}

From \file{scf\_input.c}:
\begin{verbatim}

   errcod = ip_string("LABEL",&alabel,0);
   if(errcod == IPE_OK) fprintf(outfile,"  label       = %s\n",alabel);

   reordr = 0;    /* easy to set default - if not specified, then */
                  /* this line changes nothing */
   errcod = ip_boolean("REORDER",&reordr,0); 
   if(reordr) {
      errcod = ip_count("MOORDER",&size,0);
      for(i=0; i < size ; i++) {
         errcod = ip_data("MOORDER","%d",&iorder[i],1,i);
         errchk(errcod,"MOORDER");
         }
      }
   second_root = 0;
   if (twocon) {
      errcod = ip_boolean("SECOND_ROOT",&second_root,0);
      }

   if(iopen) {
      errcod = ip_count("SOCC",&size,0);
      if(errcod == IPE_OK && size != num_ir) {
         fprintf(outfile,"\n SOCC array is the wrong size\n");
         fprintf(outfile," is %d, should be %d\n",size,num_ir);
         exit(size);
         }
      if(errcod != IPE_OK) {
         fprintf(outfile,"\n try adding some electrons buddy!\n");
         fprintf(outfile," need SOCC\n");
         ip_print_tree(outfile,NULL);
         exit(1);
         }
\end{verbatim}



\subsection{The New Binary Input and Output System}\label{C_IO_New}
% PSI3 Programmer's Manual
%
% Binary I/O --- libpsio
%
% T. Daniel Crawford, June 2000
%

\subsubsection{The structure and philosophy of the library}

Almost all \PSIthree\ modules must exchange data with raw binary (also
called ``direct-access'') files.  However, rather than using low-level
C or Fortran functions such as \celem{read()} or \celem{write()},
\PSIthree\ uses a flexible, but fast I/O system that gives the
programmer and user control over the organization and storage of data.
Some of the features of the PSI I/O system, libpsio, include:
\begin{itemize}
\item A user-defined disk striping system in which a single binary
file may be split across several physical or logical disks.
\item A file-specific table of contents (TOC) which contains
file-global starting and ending addresses for each data item.
\item An entry-relative page/offset addressing scheme which avoids
file-global file pointers which can limit file sizes.
\end{itemize}

The TOC structure of PSI binary files provdes several advantages over
older I/O systems.  For example, data items in the TOC are identified
by keyword strings (e.g., \celem{"Nuclear Repulsion Energy"}) and the
{\em global} address of an entry is known only to the TOC itself,
never to the programmer. Hence, if the programmer wishes to read or
write an entire TOC entry, he/she is required to provide only the TOC
keyword and the entry size (in bytes) to obtain the data.
Furthermore, the TOC makes it possible to read only pieces of TOC
entries (say a single buffer of a large list of two-electron
integrals) by providing the appropriate TOC keyword, a size, and a
starting address relative to the beginning of the TOC entry. In short,
the TOC design hides all information about the global structure of the
direct access file from the programmer and allows him/her to be
concerned only with the structure of individual entries. The current
TOC is written to the end of the file when it is closed.

Thus the direct-access file itself is viewed as a series of pages,
each of which contains an identical number of bytes. The global
address of the beginning of a given entry is stored on the TOC as a
page/offset pair comprised of the starting page and byte-offset on
that page where the data reside. The entry-relative page/offset
addresses which the programmer must provide work in exactly the same
manner, but the 0/0 position is taken to be the beginning of the TOC
entry rather than the beginning of the file.

\subsubsection{The user interface}
All of the functions needed to carry out basic I/O are described in
this subsection. Proper declarations of these routines are provided by
the header file \file{psio.h}. Note that before any open/close
functions may be called, the input parsing library, libipv1 must be
initialized so that the necessary file striping information may be
read from user input, but this is hidden from the programmer in
lower-level functions.  NB, \celem{ULI} is used as an abbreviation for
\celem{unsigned long int} in the remainder of this manual.

\celem{int psio\_init(void)}: Before any files may be opened or the
basic read/write functions of libpsio may be used, the global data
needed by the library functions must be initialized using this
function.

\celem{int psio\_ipv1\_config(void)}: For the library to operate properly,
its configuration must be read from the input file or from user's {\tt .psirc} file.
This call MUST immediately follow {\em int psio\_init();}.

\celem{int psio\_done(void)}: When all interaction with the
direct-access files is complete, this function is used to free the
library's global memory.

\celem{int psio\_open(ULI unit, int status)}: Opens the direct access
file identified by \celem{unit}. The \celem{status} flag is a boolean
used to indicate if the file is new (0) or if it already exists and is
being re-opened (1). If specified in the user input file, the file
will be automatically opened as a multivolume (striped) file, and each
page of data will be read from or written to each volume in
succession.

\celem{int psio\_close(ULI unit, int keep)}: Closes a direct access
file identified by unit. The keep flag is a boolean used to indicate
if the file's volumes should be deleted (0) or retained (1) after
being closed.

\celem{int psio\_read\_entry(ULI unit, char *key, char *buffer, ULI
size)}: Used to read an entire TOC entry identified by the string
\celem{key} from \celem{unit} into the array \celem{buffer}. The
number of bytes to be read is given by \celem{size}, but this value is
only used to ensure that the read request does not exceed the end of
the entry. If the entry does not exist, an error is printed to stderr
and the program will exit.

\celem{int psio\_write\_entry(ULI unit, char *key, char *buffer, ULI
size)}: Used to write an entire TOC entry idenitified by the string
\celem{key} to \celem{unit} into the array \celem{buffer}. The number
of bytes to be written is given by \celem{size}. If the entry already
exists and its data is being overwritten, the value of size is used to
ensure that the write request does not exceed the end of the entry.

\celem{int psio\_read(ULI unit, char *key, char *buffer, ULI size,
psio\_address sadd, psio\_address *eadd)}: Used to read a fragment of
\celem{size} bytes of a given TOC entry identified by \celem{key} from
\celem{unit} into the array \celem{buffer}. The starting address is
given by the \celem{sadd} and the ending address (that is, the
entry-relative address of the next byte in the file) is returned in
\celem{*eadd}.

\celem{int psio\_write(ULI unit, char *key, char *buffer, ULI size,
psio\_address sadd, psio\_address *eadd)}: Used to write a fragment of
\celem{size} bytes of a given TOC entry identified by \celem{key} to
\celem{unit} into the array \celem{buffer}. The starting address is
given by the \celem{sadd} and the ending address (that is, the
entry-relative address of the next byte in the file) is returned in
\celem{*eadd}.

The page/offset address pairs required by the preceeding read and
write functions are supplied via variables of the data type
\celem{psio\_address}, defined by:
\begin{verbatim}
  typedef struct {
    ULI page;
    ULI offset;
  } psio_address;
\end{verbatim}
The \celem{PSIO\_ZERO} defined in a macro provides a convenient input
for the 0/0 page/offset.

\subsubsection{Manipulating the table of contents}
In addition, to the basic open/close/read/write functions described above,
the programmer also has a limited ability to directly manipulate or examine
the data in the TOC itself.

\celem{int psio\_tocprint(ULI unit, FILE *outfile)}: Prints the TOC of
\celem{unit} in a readable form to \celem{outfile}, including entry
keywords and global starting/ending addresses.  (\celem{tocprint} is
also the name of a \PSIthree\ utility module which prints a file's TOC to
stdout.)

\celem{int psio\_toclen(ULI unit, FILE *outfile)}: Returns the number
of entries in the TOC of \celem{unit}.

\celem{int psio\_tocdel(ULI unit, char *key)}: Deletes the TOC entry
corresponding to \celem{key}. NB that this function only deletes the
entry's reference from the TOC itself and does not remove the
corresponding data from the file. Hence, it is possible to introduce
data "holes" into the file.

\celem{int psio\_tocclean(ULI unit, char *key)}: Deletes the TOC entry
corresponding to \celem{key} and all subsequent entries. As with
\celem{psio\_tocdel()}, this function only deletes the entry
references from the TOC itself and does not remove the corresponding
data from the file. This function is still under construction.

\subsubsection{Using \library{libpsio.a}}
The following code illustrates the basic use of the library, as well
as when/how the \celem{psio\_init()}, \celem{psio\_ipv1\_config()}, and \celem{psio\_done()}
functions should be called in relation to initialization of
\library{libipv1}.  (See section \ref{C_Program} later in the manual
for a description of the basic elements of C-language \PSIthree\
program.)
\begin{verbatim}
#include <stdio.h>
#include <libipv1/ip_lib.h>
#include <libpsio/psio.h>
#include <libciomr/libciomr.h>

FILE *infile, *outfile;

int main()
{
  int i, M, N;
  double enuc, *some_data;
  psio_address next;  /* Special page/offset structure */

  ffile(&infile,"input.dat",2);
  ffile(&outfile,"output.dat",1);
  ip_set_uppercase(1);
  ip_initialize(infile,outfile);
  ip_cwk_add(":DEFAULT");
  ip_cwk_add(progid);

  /* Initialize the I/O system */
  psio_init(); psio_ipv1_config();

  /* Open the file and write an energy */
  psio_open(31, PSIO_OPEN_NEW);
  enuc = 12.3456789;
  psio_write_entry(31, "Nuclear Repulsion Energy", (char *) &enuc,
                   sizeof(double));
  psio_close(31,1);

  /* Read M rows of an MxN matrix from a file */
  some_data = init_matrix(M,N);

  psio_open(91, PSIO_OPEN_OLD);
  next = PSIO_ZERO;/* Note use of the special macro */
  for(i=0; i < M; i++)
      psio_read(91, "Some Coefficients", (char *) (some_data + i*N),
                N*sizeof(double), next, &next);
  psio_close(91,0);

  /* Close the I/O system */
  psio_done();

  ip_done();
}

char *gprgid()
{
   char *prgid = "CODE_NAME";
   return(prgid);
}
\end{verbatim}

The interface to the \PSIthree\ I/O system has been designed to mimic
that of the old \celem{wreadw()} and \celem{wwritw()} routines of
\library{libciomr} (see the next section of this manual).  The table
of contents system introduces a few complications that users of the
library should be aware of:
\begin{itemize}
\item As pointed out earlier, deletion of TOC entries is allowed using
\celem{psio\_tocdel()} and \celem{psio\_tocclean()}. However, since
only the TOC reference is removed from the file and the corresponding
data is not, a data hole will be left in the file if the deleted entry
was not the last one in the TOC. A utility function designed to
"defrag" a PSI file may become necessary if such holes ever present a
problem.
\item One may append data to an existing TOC entry by simply writing
beyond the entry's current boundary; the ending address data in the
TOC will be updated automatically. However, no safety measures have
been implemented to prevent one from overwriting data in a subsequent
entry thereby corrupting the TOC. This feature/bug remains because (1)
it is possible that such error checking functions may slow the I/O
codes significantly; (2) it may be occasionally desirable to overwrite
exiting data, regardless of its effect on the TOC. Eventually a
utility function which checks the validity of the TOC may be needed if
this becomes a problem, particularly for debugging purposes.
\end{itemize}




\subsection{The Old Binary Input and Output System}\label{C_IO}
%
% PSI Programmer's Manual
%
% Binary I/O --- libciomr
%
% Daniel Crawford, 1 February 1996
% Updated, June 2000
%
For completeness and reference, the old PSI I/O system, libciomr, is
described here.

A binary file is identified by the \PSIthree\ module through a unit
number and not by a complete name, just as in the Fortran programming
language.  The module does not generally have access to the full name
of the physical file(s) which make up the unit --- only the low-level
I/O functions require this information.  For example, let's say the
programmer wishes to open binary file number 92 (the supermatrix file
constructed by \PSIcscf) and read data from it.  After initialization
of the input parsing system (see sections \ref{C_IP} and
\ref{PSI_Module}), he or she would call the \celem{rfile()} routine.
This function requires only the unit number of the file as an argument
--- in this case, 92.  This unit number is passed to a lower-level
routine, \celem{ioopen{\_}()}, which determines the I/O method
available.\footnote{At present, the only method available is
sequential I/O, though the originial authors of the I/O routines left
open the possibility of other, more unusual I/O techniques, including
RAM disks and asynchronous access.}  Then, the appropriate
file-opening routine is called.  This routine
(e.g. \celem{sequential{\_}ioopen()}) determines the number of volumes
(i.e. the number of physical files) across which the binary file will
be partitioned.  It then constructs the name of each physical file
based on the information provided by the user input (or, if no input
is available, a default name), and finally opens each physical file.
All of the steps beyond the call to \celem{rfile()} are conveniently
hidden from the programmer.  After the module is finished with its
interaction with the unit, the file is closed using \celem{rclose()},
which will delete the file if the programmer wishes.

There are two primary functions in \celem{libciomr.a} which allow the
PSI C modules to interact with binary files.  These are
\celem{wreadw()} and \celem{wwritw()}.  Both of these routines require
as arguments the unit number, a data buffer, the number of bytes to be
read or written, and the starting byte address in the file.  In
addition, both provide (as an argument, not a return value) the ending
bytewise file pointer after the read/write has completed.  (See
\file{libciomr/libciomr.h} for the exact syntax for calling these two
routines.)  For example, if the programmer wishes to read 512 double
precision floating point words from \FILE{92}, starting at byte number
13, into the array \celem{arr}, the appropriate call to
\celem{wreadw()} would be
\begin{verbatim}
        wreadw(92, (char *) arr, 512*sizeof(double), 13, &next_byte);
\end{verbatim}
The cast, \celem{(char *)}, is necessary so that the data in \FILE{92}
can be loaded into arrays of different types, e.g.~integer or double
precision floating point words.  Note also that a pointer to
\celem{next{\_}byte} must be passed so that the ending bytewise file
pointer will be returned.  A similar call is used for
\celem{wwritw()}.  The read/write requests are carried out by the
low-level I/O routines, which pass the data to or from the physical
files in blocks of 8192 bytes (this value may be changed by user
input).  This can significantly reduce the amount of time spent by the
CPU waiting to send or receive data to or from the physical devices.

There are four other routines which are sometimes used for reading and
writing data in binary files: \celem{sread()}, \celem{swrit()},
\celem{rread()}, and \celem{rwrit()}.  These functions work similarly
to \celem{wreadw()} and \celem{wwritw()}, but do not require bytewise
file addresses as arguments.  Instead, the global file pointer is
maintained by the I/O routines themselves, and read/write requests
automatically begin at the boundaries of so-called {\em sectors},
which are here defined to be blocks of 1024 4-byte integer words.  The
routines are frequently used by older PSI Fortran modules, which were
written before computer operating systems automatically buffered disk
I/O in memory.  However, modern workstations have made these functions
mostly obsolete, and we recommend using them only for interaction with
files constructed by older modules.


\subsection{The Structure of a \PSIthree\ C Program}\label{C_Program}
%
% PSI Programmer's Manual
%
% Essentials of a C Program
%
% David Sherrill, 31 January 1996
% Updates by TDC, 2002.
%

To function as part of the PSI package, a program must incorporate
certain required elements.  This section will discuss the header
files, global variables, and functions required to integrate a C
program into \PSIthree.  Figure \ref{fig:Essential_C_Program} presents
a minimal \PSIthree\ program, whose elements are described below.

\begin{figure}
\begin{verbatim}
                #include <stdio.h>
                #include <libipv1/ip_lib.h>
                #include <libpsio/psio.h>
                #include <libciomr/libciomr.h>

                FILE *infile, *outfile;
                char *psi_file_prefix;

                int main(int argc, char *argv[])
                {
                  extern char *gprgid(void);

                  psi_start(argc-1, argv+1, 0);
                  ip_cwk_add(gprgid());
                  psio_init();

                  /* to start timing, tstart(outfile); */
                
                  /* Insert code here */

                  /* to end timing, tstop(outfile); */

                  psio_done();
                  psi_stop();
                }

                char *gprgid(void)
                {
                   /* YOU NEED THE COLON IN THE STRING BELOW */
                   char *prgid = ":CODE_NAME";
                   return(prgid);
                }               
\end{verbatim}
\caption{The essential elements of a \PSIthree\ C-language program.}
\label{fig:Essential_C_Program}
\end{figure}

The required include files are \file{libipv1/ip\_lib.h},
\file{libciomr/libciomr.h}, \file{libpsio/psio.h}, and of course
\file{stdio.h}.  The first of these is for the Input Parser Library,
Version 1 (\file{libipv1.a}), which is described in section
\ref{C_IP}.  The second file contains function prototypes for the C
Math Routines and old-style I/O library, \file{libciomr.a}.  The third
file analogously provides clean interface to functions of the new C
I/O system described in section \ref{C_IO_New}.  The PSI libraries
require that \celem{infile}, \celem{outfile}, and
\celem{psi\_file\_prefix} be global variables.  

The integer function \celem{main()} must be able to handle
command-line arguments required by the \PSIthree\ libraries.  In
particular, all \PSIthree\ modules must be able to pass to the
function \celem{psi\_start()} arguments for the user's input and
output filenames, as well as a global file prefix to be used for
naming standard binary and text data files.  (NB: the default names
for user input and output are \inputdat\ and \outputdat, respectively,
though any name may be used.) The current standard for command-line
arguments is for all module-specific arguments ({\em e.g.},
\celem{--quiet}, used in \module{detci}) {\em before} the input,
output, and prefix values.  The \celem{psi\_start()} function expect
to find {\em only} these last three arguments at most, so the
programmer should pass as \celem{argv[]} the pointer to the first
non-module-specific argument.  The above example is appropriate for a
\PSIthree\ module that requires no command-line arguments apart from
the input/output/prefix globals.  See the \PSIthree\ modules
\module{input} and \module{detci} for more sophisticated examples.
The final argument to \celem{psi\_start()} is an integer whose value
indicates whether the output file should be overwitten (1) or appended
(0).  Most \PSIthree\ modules should choose to append.

The \celem{psi\_start()} function initializes the user's input and
output files and sets the global variables \celem{infile},
\celem{outfile}, and \celem{psi\_file\_prefix}, based on (in order of
priority) the above command-line arguments or the environmental
variables \celem{PSI\_INPUT}, \celem{PSI\_OUTPUT}, and
\celem{PSI\_PREFIX}.  The value of the global file prefix can also be
specified in the user's input file.  The \celem{psi\_start()} function
will also initialize the input parser and sets up a default keyword
tree (described in detail in section \ref{C_IP}).  This step is
required even if the program will not do any input parsing, because
some of the functionality of the input parser is assumed by
\library{libciomr.a} and \library{libpsio.a}.  For instance, opening a
binary file via \celem{psio\_open()} (see section \ref{C_IO_New})
requires parsing the \keyword{files} section of the user's input so
that a unit number (e.g.~52) can be translated into a filename.

The \celem{psi\_stop()} function shuts down the input parser and closes
the user's input and output files.

Timing information (when the program starts and stops, and how much
user, system, and wall-clock time it requires) can be printed to the
output file by adding calls to \celem{tstart()} and \celem{tstop()}
(from \library{libciomr.a}).

The sole purpose of the simple function \celem{gprgid()} is to provide
the input parser a means to determine the name of the current program.
This allows the input parser to add the name of the program to the
input parsing keyword tree.  This function is used by
\library{libpsio.a}, though the functionality it provides is rarely
used.

NB: The library \library{libciomr.a} contains older I/O functions that
have been superceded by functions in \library{libpsio.a}.  However,
you are encouraged to use the many non-I/O functions in
\library{libciomr.a}.

