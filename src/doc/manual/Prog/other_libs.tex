%
% PSI Programmer's Manual
%
% Other PSI C Libraries
%
% David Sherrill, 1 February 1996
%
There are several other PSI C libraries besides the previously-mentioned
\library{libipv1.a}, \library{libpsio.a}, and \library{libciomr.a}.
The following is a brief description of each:

??Another important library is {\tt libmalloc.a}, which provides improved memory allocation
routines. During our development of the two-electron integral
transformation program {\tt transqt}, we discovered that without
linking the library {\tt libmalloc.a}, the regular {\tt malloc()}
function allocated memory which was {\em not really} freed up when
a {\tt free()} was issued.??

\begin{description}
\item\library{libfile30.a} This library provides many routines for reading 
from and writing to the ``checkpoint'' file, \FILE{30}.  There is
generally a different function associated with each quantity in \FILE{30}
(such as the SCF energy, nuclear repulsion energy, geometry,
basis set information, etc).

\item\library{libiwl.a} The new format for storing two-electron integrals is
IWL, or ``integrals with labels.''  The library \library{libiwl.a} provides functions for
reading and writing files in the IWL format.  The code was written with 
the goal that it could be easily modified to allow for more than 256
basis functions. Its current limit is 32768 basis functions.

\item\library{libqt.a} This is the ``Quantum Trio'' library, which contains a number
of very experimental functions or functions which don't otherwise
fit anywhere else.
\end{description}

Further documentation for these libraries can be found in the \file{docs}
directory of \PSIthree\ installation.
